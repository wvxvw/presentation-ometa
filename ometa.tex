% Created 2016-05-22 Sun 09:59
\documentclass[presentation,10pt]{beamer}
\usepackage[utf8]{inputenc}
\usepackage[T1]{fontenc}
\usepackage{fixltx2e}
\usepackage{graphicx}
\usepackage{longtable}
\usepackage{float}
\usepackage{wrapfig}
\usepackage{rotating}
\usepackage[normalem]{ulem}
\usepackage{amsmath}
\usepackage{textcomp}
\usepackage{marvosym}
\usepackage{wasysym}
\usepackage{amssymb}
\usepackage{capt-of}
\usepackage[hidelinks]{hyperref}
\tolerance=1000
\usepackage[utf8]{inputenc}
\usetheme{Bergen}
\usecolortheme{spruce}
\usefonttheme{structurebold}
\author{Oleg Sivokon}
\date{\textit{<2016-05-22 Sun>}}
\title{OMeta Language}
\begin{document}

\maketitle
\begin{frame}{Outline}
\tableofcontents
\end{frame}


\section{Why?}
\label{sec-1}

\begin{frame}[label=sec-1-1]{Dedication}
\begin{quote}
Yes, but what is GlooP? If FlooP is BlooP unchained, then GlooP must
be FlooP unchained. But how can you take the chains off twice? How
do you make a language whose power transcends that of FlooP?

-- \alert{Douglas R. Hofstadter. Gödel, Esher, Bach: An Eternal Golden Braid}
\end{quote}
\end{frame}

\begin{frame}[label=sec-1-2]{Goals}
\begin{itemize}
\item Short and concise description of a program.
\item Minimalistic framework for writing compilers.
\item Extending existing languages.
\end{itemize}
\end{frame}

\section{How?}
\label{sec-2}

\begin{frame}[fragile,label=sec-2-1]{Schema}
 \begin{center}
\begin{tabular}{lll}
name & matcher and parameters & translation\\
\hline
num ::= & \verb~(0/1/2/3/4/5/6/7/8/9)*~:n & n\\
sum ::= & <num>:a \verb~+~ <num>:b => & (+ a b)\\
\end{tabular}
\end{center}
\end{frame}

\begin{frame}[fragile,label=sec-2-2]{Implementation}
 \begin{itemize}
\item Unambiguous parsing (PEG vs LR/LL).
\item Grammars as objects: inheritance, composition.
\item Parametrized rules.
\item Rules with free-style handlers (vs \texttt{visitor} pattern).
\end{itemize}
\end{frame}

\begin{frame}[label=sec-2-3]{Compared to macros}
\begin{itemize}
\item Hygiene vs scope management.
\item Structured reader-macros.
\item No need for symbol-macors and macrolets.
\end{itemize}
\end{frame}

\begin{frame}[label=sec-2-4]{Compared to compiler compilers}
\begin{itemize}
\item Full access to the underlying programming language.
\item No need for lexers.
\item No shift-reduce conflicts.
\item Meta rules.
\end{itemize}
\end{frame}

\begin{frame}[label=sec-2-5]{Compared to DCG (Prolog)}
\begin{itemize}
\item Polymorphism due to OO features.
\item Meta-rules in Prolog may be more expressive.
\end{itemize}
\end{frame}

\section{Who?}
\label{sec-3}

\begin{frame}[label=sec-3-1]{Credits and links}
\begin{itemize}
\item Developed at \href{http://vpri.org/index.html}{Viewpoints Research Institute} by Alessandro Warth
and Ian Piumarta.
\item Original implementation in SmallTalk.
\item Implementations for experimental languages like Cola.
\end{itemize}
\end{frame}

\begin{frame}[label=sec-3-2]{Industrial implementations}
\begin{description}
\item[{C\#}] \href{http://ometasharp.codeplex.com/Wikipage}{ometasharp.codeplex.com/Wikipage}.
\item[{C\#}] \href{http://ironmeta.sourceforge.net/}{ironmeta.sourceforge.net/}.
\item[{Python}] \href{https://launchpad.net/pymeta}{launchpad.net/pymeta}
\item[{Python}] \href{http://www.allbuttonspressed.com/projects/pymeta}{www.allbuttonspressed.com/projects/pymeta}
\item[{Ruby}] \href{http://github.com/aquasync/ruby-ometa/tree/master}{github.com/aquasync/ruby-ometa}
\item[{JavaScript}] \href{https://github.com/alexwarth/ometa-js}{github.com/alexwarth/ometa-js}
\item[{Scheme}] \href{http://www.lshift.net/blog/2008/07/01/ometa-for-scheme}{www.lshift.net/blog/2008/07/01/ometa-for-scheme}
\item[{Common Lisp}] \href{http://subvert-the-dominant-paradigm.net/blog/?p\%3D23}{subvert-the-dominant-paradigm.net/blog/?p=23}
\item[{Clojure (not quite)}] \href{https://github.com/ericnormand/squarepeg}{github.com/ericnormand/squarepeg}
\end{description}
\end{frame}
% Emacs 25.1.50.2 (Org mode 8.2.10)
\end{document}